\def\year{2018}\relax
%File: formatting-instruction.tex
\documentclass[letterpaper]{article} %DO NOT CHANGE THIS
\usepackage{aaai18}  %Required
\usepackage{times}  %Required
\usepackage{helvet}  %Required
\usepackage{courier}  %Required
\usepackage{url}  %Required
\usepackage{graphicx}  %Required
\frenchspacing  %Required
\setlength{\pdfpagewidth}{8.5in}  %Required
\setlength{\pdfpageheight}{11in}  %Required

\usepackage{subcaption}
\usepackage{booktabs}
\usepackage{mathtools}
\usepackage[flushleft]{threeparttable}

\graphicspath{ {./figs} }

\DeclarePairedDelimiterX{\infdivx}[2]{(}{)}{%
  #1\;\delimsize\|\;#2%
}
\newcommand{\infdiv}{D\infdivx}
\DeclarePairedDelimiter{\norm}{\lVert}{\rVert}

%PDF Info Is Required:
  \pdfinfo{
/Title (Does Novelty Disturb and Repel? Linguistic features and the Success of Fanfictions in Fandoms)
/Author (Elise Jing, Yong-Yeol Ahn, Simon Dedeo)}
\setcounter{secnumdepth}{0}  
 \begin{document}
% The file aaai.sty is the style file for AAAI Press 
% proceedings, working notes, and technical reports.
%
\title{Does Novelty Disturb and Repel? Linguistic features and the Success of Fanfictions in Fandoms}
\author{Elise Jing, Yong-Yeol Ahn, Simon Dedeo\\
}
\maketitle
\begin{abstract}
Finding out what people enjoy is crucial for the creative industry. Researches suggested that successful cultural products balance novelty and conventionality: they provide something comfortable, with a little surprise. Using a large dataset of a specific type of cultural products --- fanfictions, we study the relationships between the recognition and the novelty of a fanfiction. We quantify the novelty of a fanfiction using a term-based language model as well as a topic model, in the context of existing fanfictions in the fandom. We found that both high word-level novelty and high topic-level novelty are associated with poorer recognition among readers, diverging from the traditional theory of hedonic values of novelty.
\end{abstract}

\noindent 
What do people like? This is the central puzzle for the creative industry, a multi-billion dollar sector that produces mumerous cultural products in the form of movies, TV shows, video games, etc \cite{creativeindustries}. On one hand, novelty is central to creative works, not only because \emph{creative} works are novel by definition, but also because people seek surprise when consuming creative works \cite{hutter2011infinite}. But on the other hand, studies have shown that people also have a strong preference for familiarity. For example, when people listen to music, they often choose songs that they are familiar with \cite{thompson2014shazam}. A mere exposure to a certain stimuli can increase people's preference for it \cite{zajonc1968attitudinal}, even when they are not aware of the exposure \cite{kunst1980affective}. The exposure effect is assumed to have an evolutionary basis, and has been verified by multiple experiments \cite{bornstein1989exposure}.

Leveraging the familiarity effect is therefore a core strategy to achieve popularity. Repetition has long been utilized in cultural products including music and poetry \cite{huron2013psychological}. This trend can be also found in the contemporary popular culture, where many new releases are adaptations, remakes, and remixes of existing works \cite{manovich2007comes}. For example, Marvel and DC Comics have been making movies based on their popular comics, including many sequels, prequels, reboots, and soft reboots, many of which produced massive successes. In an extreme case, the same origin story of the Spider-man has been re-played in three movies since 2002 \cite{spiderman}. Market reception indicates that the audience enjoy such repetitions: out of the 10 highest-grossing movies of 2016, 8 are adaptions, sequels, remakes, or parts of a movie universe \cite{2016film}. This number increased to 10 out of 10 in 2017 \cite{2017film}. 

As a reconciliation for the apparent conflict between the preference for novelty and familiarity, the \emph{optimal differentiation hypothesis} \cite{thompson2017hit} has been widely adopted. It states that successful creative works are combinations of convention and innovation; the most popular ones are differentiated from previous works and their peers, but not \emph{too} different. In psychology, this is captured in the Wundt-Berlyne curve, which suggests that when exposed to novel stimuli, a perceiver's positive feelings first increase as novelty increases, until reaching a certain threshold; afterwards, further increasing novelty will lead to a decrease in positive feelings \cite{berlyne1970novelty}. Experiments have supported this hypothesis \cite{hargreaves1984effects} \cite{sluckin1980liking}, which has been used to study the success of a variety of cultural products. For example, Askin and Mauskapf found that songs with optimal differentiation are more likely to be on the top of the Billboad's Hot 100 charts \cite{askin2017makes}. Sreenivasan found a similar pattern for films \cite{sreenivasan2013quantitative}, and Mukherjee et al. found that in scientific publications, the highest-cited papers are grounded on mostly conventional, but partly novel combinations of previous works. 

However, most of these researches rely on external metrics to measure the success of creative works, such as the box office or  the Billboard chart. These are highly influenced by factors not directly relevant to a creative works' quality, such as advertisement and media coverage. In particular, the consumers' choices are largely shaped by the herding behavior. For example, it was found that being on the New York Times bestseller book list will cause an increase in sales \cite{sorensen2007bestseller}.  Even a ``faked'' popularity can be turned into real popularity as people are more likely to view creative works that they perceive to be popular \cite{salganik2008leading}. Therefore, such metrics may not accurately reflect the consumers' actual enjoyment of a cultural product. Moreover, as the features of cultural products vary significantly across genres, lengths, subjects, etc, it is hard to measure the novelty or conventionality of one piece against the others. 

Here, we instead study a special type of creative works that are directly created and consumed by fans --- fan works. Known formally as transformative works, they are creative works made by fans based on one or more original works (``canons'') \cite{wiki:transf_work}. For example, a story written by a contemporary fan about Sherlock Holmes in his retirement is considered a fan work. Although fan works can take multiple media types such as paiting, music and games, the most common type is creative writing: fanfictions. As a unique cultural phenomena, fanfictions have drawn attention from media studies and cultural studies \cite{thomas2011fanfiction}. However, most of the existing studies focus on the identity of fanfiction writers \cite{black2006language}, the practice of fanfiction writing \cite{LIT:LIT12061}, and the interaction between fans \cite{hills2015expertise}, and it is difficult to find studies that focus on fanfictions themselves as creative works.

Using fanfictions as subjects of study, we may avoid some of the complications mentioned above. The success of a fanfiction can be measured more directly with less influence from other media. In the service that we used to collect data, readers can give ``kudos'' to each work, expressing that they enjoyed reading it, as well as leaving comments and bookmarks. As a subculture, fanfictions are usually not published as books nor featured in mainstream media, and there is no ``bestseller list'' for fanfictions, reducing the effect of the herding behavior. Additionally, most fanfictions are freely available in the Internet, so that the price is not a confounding factor. Moreover, because fans of a canon usually form communities known as fandoms \cite{wiki:fandom}, sharing the same context, fanfictions have less variation in subjects. Finally, because they are freely available as plain texts without the copyright issues, natural language processing methods can be readily applicable to operationalize novelty quantitatively.

With these advantages, we investigate how a fanfiction's novelty or familiarity influences its success. We use two methods to estimate the novelty of the fanfictions: one based on the word features, and one on the topic features, with respect to other fanfictions in the same fandom. A fanfiction is more novel if it has a large distance to other fanfictions published during the same time period in the feature space, and vice versa. Using statistical models, we found a negative relationship between a fanfiction's novelty and its success, consistent across both methods, suggesting that in disproval of the optimal differentiation hypothesis, fanfiction readers prefer familiarity and discourage novelty in general.


\section{Results} 
We collect data from an online fanfiction archive, Archive of Our Own (AO3), which allows users to upload their fanfictions and categorizes them based on fandoms. Established in 2009, it has become one of the largest fan communities, with 1,313,000 users and 3,423,000 pieces of fanfictions by November 2017 \cite{ao3stats}.

AO3 classifies the fandoms based on formats, such as movies, TV shows, books, anime \& manga, and musicals. We first list the top 5 fandoms in each category based on the number of fanfictions. Then we exclude the fandoms that are subsets of other large fandoms: for example, we keep \emph{Marvel} but exclude \emph{The Avengers}, since the fanfictions in \emph{The Avengers} are also in the \emph{Marvel} fandom. Fandoms that cover diverse topics such as \emph{k-pop} are also removed. We only kept the fictions written in English. Finally, to control for the effects of fiction length on the readers' preference and on our methods, we only analyze the fictions with 500 - 1500 words in each chapter, which is the range where most fictions lie in(see Supp). This leaves us with 701,635 fanfictions in 24 fandoms. 


\subsection*{Descriptive statistics}

Figure \ref{fig:fandom_size} shows the number of fanfictions in each of the 24 fandoms that we study. The fandoms with the largest amounts of fictions are \emph{Marvel}, \emph{Supernatural}, and \emph{Sherlock Holmes}. Figure \ref{fig:fandom_size}  shows the time series of the fanfictions' volume. As AO3 was established in December 2009, the fictions earlier than this time might be migrated from other platforms, and may not correctly reflect the status of the archive. We therefore only run our analysis using the fictions published in January 2010 or later. After its establishment, the site first experienced a slow growth until around 2012, when the number of fictions published begin to rise rapidly. 


\begin{figure*}
    \centering
    \begin{subfigure}[b]{\textwidth}
    \hspace{2cm}
        \includegraphics[width=0.7\textwidth]{/fic_time_fandom_size_dist.pdf}
        \caption{The size of fandoms and the number of fanfictions published in time.}
        \label{fig:fandom_size}
    \end{subfigure}
    ~ %add desired spacing between images, e. g. ~, \quad, \qquad, \hfill etc. 
      %(or a blank line to force the subfigure onto a new line)
    \begin{subfigure}[b]{\textwidth}
            \includegraphics[width=0.9\textwidth]{/kudos_comments_hits_bookmarks_ccdf.pdf}
        \caption{Log-lin complementary cumulative distribution of kudos, hits, bookmarks and comments. For multi-chapter fanfictions, we average these values over the number of chapters. Fat-tailed distributions are observed, where a small portion of fictions receive many kudos and comments, and most fictions receive few.}
        \label{fig:kudos_dist}
    \end{subfigure}
\end{figure*}


\begin{table*}
\centering
\begin{tabular}[width=0.8\textwidth]{p{3cm}p{10cm}}
\toprule
Fields & Description \\ 
   \hline			
Title & Titles of the fanfictions.  \\
Fandoms & The fandom(s) that the fanfiction belongs to \\
Author & The author of the fanfiction.  \\
Chapters & The number of chapters that a fanfiction have. \\
Archive Warnings & The warnings for sensitive elements in the fanfiction. \\
Category & The type of relationships in the fanfiction. \\
Rating & The age rating of the fanfiction . \\
Relationship & The relationship(s) between characters in the fanfiction, in the form of Character A/Character B
or Character A\&Character B. \\
Publish Date & The date the fiction was published \\
Complete date & For serial fanfictions, the date when it was marked as ``complete''.\\
Update date & The date when the fanfiction was last updated. \\
\hline
Kudos & The number of kudos (likes) that the fanfiction received. \\
Comments & The number of comments that a fanfiction received.\\
Hits & The number of times a fanfiction is clicked on. \\
Bookmarks & The number of people who bookmarked the fanfiction.\\

\bottomrule
\end{tabular}
\caption{Metadata fields of the fanfictions.}
\label{tab:metadata}
\end{table*}%

We also collected metadata for each fanficiton (see Table \ref{tab:metadata}). Some of them can be used to evaluate the success of the fanfictions. While kudos is the most direct indication, being a straightforward expression that a reader liked the fiction, the hits is also an indicator of popularity. A reader may choose to bookmark a fanfiction to read later. This may depend on multiple motivations, such as the length of the fiction and the time when the reader browses the website; we also do not know if the reader returns to read the fiction later. The number of comments shows the engagement of readers with a fanfiction, and is indirectly associated with popularity. Based on these considerations, we choose to use kudos as the primary indicators of success, and hits, comments and bookmarks as secondary indicators. The correlation between these indicators can be found in Figure \ref{fig:corr}. Figure \ref{fig:kudos_dist} shows the cumulative distribution of kudos and comments in a logarithmic scale. Following fat-tailed distributions, a small number of fictions receive the majority of kudos and comments, while most fictions receive little or none. 

While we focus on how a fanfiction's novelty influences its success, there are other factors at play. For example, some fandoms have larger fan bases than others, resulting in a high average Kudos for the fanfictions in these fandoms. The popular authors' fame may influence how their works are received. Since AO3's user base has been increasing, we may expect newer fanfictions to receive more Kudos than older ones. Multi-chapter fanfictions may also be likely to receive more Kudos than single-chapter ones. The ``Archive Warnings'' indicate that the fanfictions contain sensitive elements such as graphic violence or major character death, and may influence the readers' choice to read them. Other intrinsic features, such as the ratings of the fictions and the relationships in them may also influence the readers' preferences. In subsequent analysis, we use multiple strategies to control for such influences.


\subsection*{Fanfictions' Novelty and Success}
Following the previous approaches \cite{askin2017makes} \cite{de2015game}, we evaluate a fanfiction's novelty by comparing it to other previously published fanfictions. Intuitively, if a fanfiction is close to the center of a feature space containing all past fictions, it is less novel; reversely, a peripheral position indicates higher novelty. 

We extract features from the fanfictions using two methods. The first one is a vector space model as often used in information retrieval \cite{turney2010frequency}, where each fanfiction is represented as a vector, its entries being the term frequency-inversed document frequency (TF-IDF) score of the unique words in the fiction. This allows us to identify the most representative terms in the documents, quantifying the term-level novelty. The feature space is created in a way that for each fanfiction, it contains all fanfictions published in the same fandom within the past 6 months before it was published. We then compute the center of the feature space as the average of all feature vectors. The \emph{term novelty} score of a fanfiction is defined as the cosine distance between its vector representation and the center:

\begin{equation}
N = 1-\frac{\sum {f\cdot{c}}}{\sqrt{\sum{\norm{f}^2}}\sqrt{\sum{\norm{c}^2}}}
\end{equation}

where $f$ is the vector representation of a fanfiction and $c$ is the center of the vector space.

Alternatively, we also use the \emph{Latent Dirichlet Allocation} (LDA) \cite{blei2003latent} to model the topics of the fanficitions, quantifying their novelty on the topic level. As a widely used topic modeling method, LDA models a fiction as a probabilistic distribution over a set of topics. Similar as in the previous method, for each fanfiction we construct a feature space consisting of the topic distributions of the fanfictions published before it, and define the \emph{topic novelty} score as the cosine distance between the fanfiction's topic distribution and the center. 

Figure \ref{fig:tfidf_lda_kudos} plots the relationship between the fanfictions' novelty and the kudos they receive, aggregated across fandoms. Since the fandoms differ in reader base, we compute the z-score for each value of kudos based on the average number of kudos per fanfiction in its fandom. We observe that as the term novelty and topic novelty score increases, the z-score of Kudos steadily decrease in both cases except for one outlier. The less novel fanfictions receive more than average kudos, and the highly novel fanfictions receive less than average. Similar patterns can be observed between novelty and hits, comments and bookmarks (See supplementary material).

\begin{figure*}
    \centering
          \includegraphics[width=0.8\textwidth]{/novelty_kudos_tfidf_lda.pdf}
        \caption{The relationship between novelty and kudos. Left: term novelty. Right: topic novelty. The horizontal axes are the novelty scores, and the vertical axes are the corresponding average of the z-score of kudos in bins with bin size = 0.1. The confidence intervals obtained from bootstrap resampling are shown. }
        \label{fig:tfidf_lda_kudos}
\end{figure*}


We perform regression analysis to further explore this relationship. Four models are created with response variables being the logarithm of kudos, hits, comments, and bookmarks respectively, and the term novelty and topic novelty scores being the predictor variables in all models. Control variables are added to account for the influence of other factors on the reception of fanfictions. First, the number of chapters that a fanfiction has is included, considering that multi-chapter fictions have more chances of exposure. Secondly, categorical variables are created to capture the effects of the fandoms, archive warnings, categories, and ratings. Thirdly, the authors may accumulate fame by writing more fanfictions. We therefore use a binary variable \emph{frequent author} to indicate whether an author has written more than 10 fanfictions. Fourthly, the relationship between characters is a main reason that fans read fanfictions, and some relationships have a larger fan base than others. To account for this effect, we create another binary variable \emph{frequent relationship} to capture whether a fanfiction features a relationship that is among the top 5 most frequent relationships in the fandom. Finally, we include the age of each fanfiction as a numerical variable, which is the number of days since a fanfiction was last edited. The correlations between the numerical variables are shown in Figure \ref{fig:corr}, showing no strong correlation between the predictor variables.

Because of the excessive zeroes in the response variables, we apply a Heckman--style correction with two steps. A Logistic regression is first performed on the predictor and control variables to predict the probability of each sample having a non-zero outcome. This probability is then used as an additional predictor variable in a pooled OLS regression for each model.

\begin{figure}
    \centering
          \includegraphics[width=0.5\textwidth]{/variables_corr.pdf}
        \caption{Correlations between the predictor and outcome variables. Strong collinearity exists between the outcome variables, but not the predictor variables. }
        \label{fig:corr}
\end{figure}

Selected OLS coefficient estimates of each model are summarized in Table \ref{tab:regression} (for the full coefficients table, see Supplementary materials). We notice that a fanfiction featuring frequent relationships and published in certain fandoms are likely to have better reception. Meanwhile, elements such as character death and mature contents are not beneficial to their reception. In contrast to our assumption, the number of chapters, the author's fame, and the age do not have significant influence on the fanfictions' success. After controlling for these effects, we find that the term novelty and topic novelty both have significant negative coefficient estimates. This further supports our previous observation that higher novelty contributes to a fanfiction's receiving less kudos, hits, comments, and bookmarks. As the reception of a fanfiction is shaped by many factors including possible noise, we consider the signals found with this model to be a strong indication.

 \begin{table*}
\centering
\begin{tabular}[width=0.8\textwidth]{p{4cm}p{3cm}p{3cm}p{3cm}p{3cm}}
\toprule
Response variable: & Kudos & Hits & Comments & Bookmarks \\ 
   \hline			
Term novelty & -0.5961***&  -0.4339***  & -0.1794*** & -0.6326*** \\
& (0.008) & ( 0.012) & (0.020) & (0.014)\\
Topic novelty & -2.3696*** & -1.4229*** & -0.9851*** & -4.0823*** \\
& (0.047) & (0.060) &  (0.026 ) & (0.077)\\
Chapters & -0.0072***& -0.0043*** & 0.0002***& -0.0042*** \\
& (0.000) & (4.88e-05) & (8.64e-05) & ( 0.000) \\
Frequent relationship & 0.3452*** & 0.4332*** & 0.1657*** & 0.6386***\\
& (0.006) & (0.007) & (0.013) & (0.010)\\
Frequent author & -0.2350*** & -0.2747*** & 0.0921*** & 0.0970***\\
& (0.005) & (0.007) & (0.004) & (0.009)\\
Category (Female/Male) & -0.4200*** & -0.1943*** & -0.1000*** & -0.3440***\\
& (0.006) & (0.007) & (0.010) & (0.009)\\
Category (Male/Male) & 0.1742*** & 0.3030*** & 0.1107*** & 0.5297***\\
& (0.006) & (0.008) & (0.011) & (0.011)\\
Archive warnings (death) & -0.3897*** & 0.2025*** & -0.1260*** & -0.4673***\\
& ( 0.010) & (0.013) & (0.010) & (0.017)\\
Rating (General audience) & 0.0656*** & -0.1002*** & -0.0961*** & -0.5963***\\
& ( 0.008) & (0.013) & (0.006) & (0.015)\\
Rating (Mature) & -0.3883*** & 0.3668*** & -0.0988*** & -0.4289***\\
& (0.008) & (0.011) & (0.009) & (0.012)\\
Fandom (Doctor Who) & -0.3653*** & 0.3405*** & -0.0857*** & -0.5748***\\
& (0.006) & 0.008 & (0.008) & (0.010)\\
Fandom (Star Wars ) & 0.3571*** & 0.1642*** & 0.2160*** & 0.2866***\\
& (0.016) & (0.020) & (0.017) & (0.027)\\
Age & 0.0001***& 0.0002*** & -9.094e-05***& 3.38e-05***\\
& (4.43e-06) & (7.79e-06) & (9.83e-06) & (7.36e-06)\\
\hline 
Observations & 526992 & 526992 & 526992 & 526992 \\
Adjusted R-squared & 0.247 & 0.115 & 0.116  & 0.150\\
\bottomrule
\end{tabular}
 \begin{tablenotes}
      \small
      \item Standard errors are in parentheses.
      \item ***p $<$ 0.001.
       \end{tablenotes}
\caption{OLS results}
\label{tab:regression}
\end{table*}%

Although we find an overall negative relation between novelty and success, this do not rule out the possibility of non-linear relationships, such as the reversed U-shape curve. We therefore turn to generalized additive models (GAM) \cite{wood2006generalized} for non-parametric estimation. The same set of predictor variables, control variables and response variables are used, except for that we use the original values of response variables instead of their logarithms. The fitted plots with kudos as response variable for term novelty and topic novelty are shown in Figure \ref{fig:gam} (for plots of other response variables, see Supplementary material). We observe an overall monotonic decrease in both cases, except for a small increase in the high topic novelty region.

\begin{figure*}
    \centering
          \includegraphics[width=0.8\textwidth]{/gam_results.pdf}
        \caption{Fit plots of the generalized additive models.}
        \label{fig:gam}
\end{figure*}

\section*{Discussion}
Traditional theories about pop culture suggest that people like things with a lot of conventionality and a little surprise. Our findings with fanfictions diverge from this pattern: people like no surprise and will always stay to the familiar. This may be the nature of fan works --- because fans desire to see familiar characters and stories, it is reasonable for them to prefer fanfictions with familiar elements. This may extend to other parts of pop culture, explaining people's desire to listening to old songs and justifying the abundance of sequel and spin-off works. However, it does not account for the emergence of novelty. Sometimes, non-stereotypical creative works enjoy great success: if people do not show a preference for innovation, they can only be attributed to a few highly innovative artists.

We have developed a method to quantify the novelty of writings by comparing its linguistic features to other writings. We also measure novelty on two levels: the word level and the topic level. Although we choose to focus on fanfictions, our method can be easily extended to other types of creative writings. A potential research direction is therefore to investigate the relationship between novelty and success in other areas, such as book markets or news articles. 



\section*{Methods}

\subsection*{Data collection}
A Python script was used to obtain data from AO3 (http://archiveofourown.org/) as csv files in March 2016. We filtered the data based on the language and length of the fanfictions, and replaced missing values where possible. Several extreme values are also removed as they are likely to be outliers.

\subsection*{Modeling fanfictions}
The Python library scikit-learn was used to create the vector space models. Another library Gensim is used to fit LDA topic models on our data. We set the number of topics to 100 and use default values for other parameters. 

\subsection*{Regression analysis}
We use the Python library statsmodels \cite{seabold2010statsmodels} to perform the pooled OLS regression. The GAM models are fitted using the pyGAM library \cite{pygam}, with parameters $splines=25$ and $lambda=301.25$.


\section{ Acknowledgments}
We thank the AO3 staff for helping us with the data collection.



\bibliography{main.bib}
\bibliographystyle{aaai}
\end{document}

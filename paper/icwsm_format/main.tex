


\def\year{2018}\relax
%File: formatting-instruction.tex
\documentclass[letterpaper]{article} %DO NOT CHANGE THIS
\usepackage{aaai18}  %Required
\usepackage{times}  %Required
\usepackage{helvet}  %Required
\usepackage{courier}  %Required
\usepackage{url}  %Required
\usepackage{graphicx}  %Required
\frenchspacing  %Required
\setlength{\pdfpagewidth}{8.5in}  %Required
\setlength{\pdfpageheight}{11in}  %Required

\usepackage{subcaption}
\usepackage{booktabs}
\usepackage{mathtools}
\usepackage[flushleft]{threeparttable}

% temporary formatting for editing
\linespread{2}
\pagestyle{plain} 

\graphicspath{ {./figs} }

\DeclarePairedDelimiterX{\infdivx}[2]{(}{)}{%
  #1\;\delimsize\|\;#2%
}
\newcommand{\infdiv}{D\infdivx}
\DeclarePairedDelimiter{\norm}{\lVert}{\rVert}

%PDF Info Is Required:
  \pdfinfo{
/Title (Does Novelty Disturb and Repel? Linguistic features and the Success of Fanfictions in Fandoms)
/Author (Elise Jing, Simon DeDeo, Yong-Yeol Ahn)}
\setcounter{secnumdepth}{0}  
 \begin{document}
% The file aaai.sty is the style file for AAAI Press 
% proceedings, working notes, and technical reports.
%
\title{Does Novelty Disturb and Repel? Linguistic features and the Success of Fanfictions in Fandoms}
\author{Elise Jing, Simon Dedeo, Yong-Yeol Ahn\\
}
\maketitle
\begin{abstract}
Finding out what people enjoy is crucial for the creative industry. The \emph{optimal differentiation hypothesis} suggests that successful cultural products balance novelty and conventionality: they provide something comfortable, with a little surprise. Using a large dataset of fanfictions, we study the relationships between the   novelty of a fanfiction and how it is recognized. We quantify the novelty of a fanfiction using a term-based language model as well as a topic model, in the context of existing fanfictions in the same fandom. We found that in general, low novelty is associated with popularity and higher novelty indicates poorer recognition. In a few cases, however, very high novelty are found to enjoy success. Diverging from the traditional theory of hedonic values of novelty, our results suggest that people may prefer creative works that are familiar, while appreciating exceptionally novel creations.
\end{abstract}

\section*{Introduction}
\noindent 
What do people like? This is the central puzzle for the arts and cultural industries, a sector that contributes more than \$700 billion to the U.S. GDP in 2018 by products in the form of movies, TV shows, video games, etc \cite{artsculturalindustries}. On one hand, novelty is central to creative works, not only because \emph{creative} works are novel by definition, but also because people may seek surprise when consuming creative works \cite{hutter2011infinite}. But on the other hand, studies have shown that people also have a strong preference for familiarity. For example, when people listen to music, they often choose songs that they are familiar with \cite{thompson2014shazam}. Humans also find high attractiveness in faces that are close to the average, and therefore more familiar. 
The effect even extends to images of other objects such as birds and automobiles \cite{Halberstadt2003}. Moreover, the mere exposure effect found by Zajonc show that exposure to a certain stimuli can increase people's preference for it \cite{zajonc1968attitudinal}, even when they are not aware of the exposure \cite{kunst1980affective} \cite{bornstein1989exposure}. 

Leveraging the familiarity effect is therefore a core strategy to achieve popularity. Repetition has long been utilized in cultural products including music and poetry \cite{huron2013psychological}. This trend can be also found in the contemporary popular culture, where many new releases are adaptations, remakes, and remixes of existing works \cite{manovich2007comes}. For example, Marvel and DC Comics have been making movies based on their popular comics, including many sequels, prequels, reboots, and soft reboots, many of which produced massive successes. In an extreme case, the origin story of the Spider-man has been re-played in three movies since 2002: \emph{Spider-Man (2002)}, \emph{The Amazing Spider-Man (2012)} and \emph{Spider-Man: Homecoming (2017)} \cite{spiderman}. Market reception indicates that the audience enjoy such repetitions: these three movies grossed \$636,480,300, \$308,489,900, and \$342,750,600 respectively (inflation-adjusted) \cite{spider-gross}, being among the top 300 highest-grossing movies of all time. Out of the 10 highest-grossing movies of 2016, 8 are adaptions, sequels, remakes, or parts of a movie universe \cite{2016film}. This number increased to 10 out of 10 in 2017 \cite{2017film}. Notably, many such films also follow the classical plots of their genres. Film and literary theories have suggested that most stories can be summarized into a few archetypes and basic elements, such as the Hero's Journey \cite{campbell2008hero}, the mythemes \cite{levi1955structural} or the Proppian functions \cite{propp2010morphology}. In this sense, every story involves the repetition of classical narrative elements.

Extremely novel cultural products, on the other hand, sometimes also enjoy huge successes. In the film industry, such movies are often landmarks in terms of cinematography techniques or story telling, such as \emph{Star Wars} (multiple ground-breaking technologies) and \emph{Avartar} (the first mainstream 3D film using stereoscopic filmmaking). In the popular music scene, new genres such as hip-hop and electronic dance music constantly emerged throughout the 20th century, starting as underground music and growing to enjoy global commercial success. In the history of literature, experimental works such as \emph{The Trial} by Franz Kafka and \emph{Ulysses} by James Joyce are considered among the most important works of all times. If people only enjoy familiar things, why are there such novel creations?

\textbf{Theory on novelty, familiarity and liking}    As a reconciliation for the apparent conflict between the preference for novelty and familiarity, the \emph{optimal differentiation} hypothesis has been widely adopted. It states that successful creative works are combinations of convention and innovation; the most popular ones are differentiated from previous works and their peers, but not \emph{too} different. In psychology, this idea is captured in the Wundt-Berlyne curve, an inverted U-shaped curve with an optimal amount of novelty for hedonic values \cite{berlyne1970novelty}. A few experiments have supported this hypothesis regarding words \cite{sluckin1980liking}, music \cite{hargreaves1984effects}  \cite{askin2017makes},  films \cite{sreenivasan2013quantitative} and scientific publications \cite{uzzi2013atypical}, suggesting that, for example, songs with optimal differentiation are more likely to be on the top of the Billboad's Hot 100 charts. Movies with optimal differentiation have highest revenue. In scientific publications, the highest-cited papers are grounded on mostly conventional, but partly novel combinations of previous works. However, most of these researches deal with cultural products whose reception are influenced by many factors, including intrinsic factors such as subject, genre, and length, as well as external ones such as price, advertisement, and media coverage. For example, deliberately putting unpopular songs on the top chart can in fact popularize them \cite{salganik2008leading}. The complex interactions between consumers, cultural products and the market is therefore difficult to disentangle. 
 
\textbf{Fanfictions as data source}   Here, we instead study a special type of creative works --- fan works, which allows us to avoid 
some of the aforementioned complications. Formally known as ``transformative works'', they are creative works made by fans based on one or more ``canons'' \cite{wiki:transf_work}. For instance, a fan of Sherlock Holmes may write a story about Holmes in his retirement. Although fan works can take multiple media types such as painting, music, and games, the most common type is creative writing: fanfictions. As a unique cultural phenomena, fanfictions have drawn attention from media studies and cultural studies \cite{thomas2011fanfiction}. However, most of the existing studies focus on the identity of fanfiction writers \cite{black2006language}, the practice of fanfiction writing \cite{LIT:LIT12061}, and the interaction between fans \cite{hills2015expertise}, rather than fanfictions themselves as creative works.

We choose to use the fanfiction data for a few reasons. First, because fanfictions are usually not published as books nor featured in mainstream media, but shared among relatively isolated online communities (``fandoms'') with almost no advertisement or promotions \cite{wiki:fandom}, the success of a fanfiction is less swayed by the social spreading processes. Secondly, most fanfictions are freely available in the Internet, so that the price is not a confounding factor. Thirdly, fanfictions in the same fandom are created using the same context, sharing similar subjects, characters, and settings. The variations thus occur mainly on how the ``ingredients'' are put together, allowing us to focus on their novelty while controlling for other factors. Finally, because they are freely available as plain texts, natural language processing methods can be readily applicable to quantitatively operationalize novelty. 

With these advantages, we investigate how a fanfiction's novelty is associated with its success. We use two frameworks to estimate the novelty of the fanfictions:  term based language model and topic modeling. In both cases, a fiction is evaluated with respect to existing fanfictions in the same fandom. A fanfiction is more novel if it is more different from the other fanfictions published during the same time period in the feature space. Using statistical models, we found an overall monotonically decreasing relationship between the fanfictions' novelty and their successes, with an uptick for very high novelty fictions in some cases. Our finding diverges from the optimal differentiation hypothesis, suggesting that in the context of fanfictions, low novelty is generally associated with better reception. Some  high novelty works may also enjoy success, albeit rarely. Readers are found to prefer creative works that give them a sense of familiarity, or those that are highly surprising.

\section*{Data \& methods}

We collect data from an online fanfiction archive, Archive of Our Own (AO3), which allows users to upload their fanfictions and categorizes them based on fandoms. Established in 2009, it has become one of the largest fan communities, with more than 1.6 million users and 4 million pieces of fanfictions by August 2018 \cite{ao3stats}. A Python script was used to download fanfictions and their metadata from AO3 (http://archiveofourown.org/) in March 2016. 

AO3 classifies the fandoms based on the formats of the canons, such as movies, TV shows, books, anime \& manga, and musicals. We first identify the top 5 fandoms in each category based on the number of fanfictions. Then we exclude the fandoms that are subsets of other large fandoms. For example, we keep \emph{Marvel} but exclude \emph{The Avengers}, because \emph{The Avengers} is a part of the Marvel universe. Fandoms that are too general, such as \emph{k-pop}, are also removed. We only kept the fictions written in English. Finally, to control for the effects of fiction length on the readers' preference and on our methods, we only analyze the fictions with 500 -- 1500 words in each chapter, which is the range where most fictions lie in. This leaves us with 701,635 fanfictions in 24 fandoms. 

Figure \ref{fig:fandom_size} (a) shows the number of fanfictions in each of the 24 fandoms that we study. The fandoms with the largest amounts of fictions are \emph{Marvel}, \emph{Supernatural}, and \emph{Sherlock Holmes}. Figure \ref{fig:fandom_size} (b) shows the time series of the fanfictions' volume. As AO3 was established in December 2009, the fictions earlier than this time might be migrated from other platforms, and may not correctly reflect the status of the archive. We therefore only run our analysis using the fictions published in January 2010 or later. After a slow growth until around 2012, the service experienced more rapid growth. 

\begin{figure*}
    \centering
   % \hspace{1cm}
        \includegraphics[width=\textwidth]{/fic_time_fandom_size_dist.pdf}
        \caption{The size of fandoms and the number of fanfictions published in time.}
        \label{fig:fandom_size}    
    \end{figure*}
    
\begin{figure*}
    \centering
       \includegraphics[width=\textwidth]{/kudos_comments_hits_bookmarks_ccdf.pdf}
        \caption{Log-log complementary cumulative distribution of kudos, hits, bookmarks and comments. For multi-chapter fanfictions, we average these values over the number of chapters. Fat-tailed distributions are observed, where a small portion of fictions receive many kudos and comments, and most fictions receive few.}
        \label{fig:kudos_dist}
    \end{figure*}

\begin{table}
\centering
\begin{tabular}[width=\textwidth]{p{3cm}p{5cm}}
\toprule
Fields & Description \\ 
   \hline			
Title & Titles of the fanfictions.  \\
Fandoms & The fandom(s) that the fanfiction belongs to. \\
Author & The author of the fanfiction.  \\
Chapters & The number of chapters that a fanfiction have. \\
Archive Warnings & The warnings for sensitive elements in the fanfiction. \\
Category & The type of relationships in the fanfiction. \\
Rating & The age rating of the fanfiction . \\
Relationship & The relationship(s) between characters in the fanfiction, in the form of Character A/Character B
or Character A\&Character B. \\
Publish Date & The date the fiction was published \\
Complete date & For multi-chapter fanfictions, the date when it was marked as ``complete''.\\
Update date & The date when the fanfiction was last updated. \\
\hline
Kudos & The number of kudos (likes) that the fanfiction received. \\
Comments & The number of comments that a fanfiction received.\\
Hits & The number of times a fanfiction is clicked on. \\
Bookmarks & The number of people who bookmarked the fanfiction.\\

\bottomrule
\end{tabular}
\caption{Metadata fields of the fanfictions.}
\label{tab:metadata}
\end{table}%

We also collected metadata for each fanfiction (see Table \ref{tab:metadata}). Variables such as kudos, hits, comments, and bookmarks are used to operationalize the success of a fanfiction. While the hits is the most direct metric for popularity, the kudos is a straightforward expression that a reader liked the fiction. A reader can also comment on a fiction or bookmark it to read later. The comments and bookmarks therefore signal the recognition or engagement from readers, although they depend on multiple motivations, and are less directly associated with popularity. Figure \ref{fig:kudos_dist} shows the cumulative distribution of these indicators on a logarithmic scale. Like many other measurements of popularity, they exhibit fat-tails. That is, a small number of fictions receive the majority of kudos, etc, while most fictions receive little or none. We use the logarithm of these values, which is a common practice for similar data such as citations. This helps to reduce the potential bias when performing regression and many other statistical analysis \cite{thelwall2014regression}. 

\textbf{Quantifying novelty}   As previous studies noted \cite{askin2017makes,de2015game}, it is reasonable to assess a fanfiction's novelty in the context of other previously published fanfictions. Intuitively, a fanfiction is less novel if it is similar to many other fictions published before. Here we use the centroid of all past fictions in a fandom in a feature space as the guide for measuring novelty. Namely, we assume that a fanfiction is less novel if it is close to the center, and vice versa. 

We extract features from the fanfictions using two methods, the term frequency - inversed document frequency (TF---IDF) and the Latent Dirichlet Allocation (LDA) \cite{blei2003latent}. TF---IDF is a vector space model often used in information retrieval \cite{salton1988term}, where each fanfiction is represented as a vector, and its entries are scores of the unique terms in the fiction. This allows us to quantify the term-level novelty. The TF-IDF scores are computed, for each fanfiction, using all fanfictions published in the same fandom within the past 6 months from when it was published. We then compute the center of the feature space as the average of all feature vectors. The \emph{term novelty} score $s_i$ of a fanfiction is defined as:

\begin{equation}
s_i = 1-\frac{\boldsymbol{f_i} \cdot{\boldsymbol{f_i^{(c)}}}} {\lvert \boldsymbol{f_i} \rvert \lvert \boldsymbol{f_i^{(c)}} \rvert}
\end{equation}

where $\boldsymbol{f_i}$ is the vector representation of a fanfiction and $\boldsymbol{f_i^{(c)}}$ is the center of the vector space defined with respect of this fanfiction.

We also use the LDA to quantify the fanfictions' novelty in terms of their topics. LDA models a document as a probabilistic distribution over a set of topics, each of which is a probability distribution across terms. Similar to the TF---IDF, for each fanfiction, we construct a feature space consisting of the topic distributions of the fanfictions published before it, and define the \emph{topic novelty} score as the cosine distance between the fanfiction's topic distribution and the center of the feature space. (See Appendix for the model details.)

\textbf{Control factors}    To isolate the relationships between the novelty and success, we need to control for other confounding factors. For example, some fandoms have larger fan bases than others, resulting in higher numbers of kudos for the fanfictions in these fandoms. The authors' fame may influence how their works are received. Since AO3's user base has been increasing, we may expect newer fanfictions to receive more kudos than older ones. Multi-chapter fanfictions may also be likely to receive more kudos than single-chapter ones. The ``Archive Warnings'' indicate that the fanfictions contain sensitive elements such as graphic violence or major character death, and may influence the readers' choice to read them. Other intrinsic features, such as the age ratings of the fictions and the relationships in them may also influence the readers' preferences. In subsequent analysis, we use multiple strategies to control for such influences.


\section{Results}

\textbf{Raw correlation between novelty and success}    Figure \ref{fig:tfidf_lda_kudos} displays the raw correlation between the fanfictions' novelty and the kudos, hits, comments, and bookmarks that they receive, aggregated across fandoms. Since the fandoms differ in size and activity, we compute the z-score for each value based on the average of its fandom. We observe that as the term novelty and topic novelty score increases, the z-score of kudos, hits, comments, and bookmarks decrease in general in both cases, with the exception of some small fluctuations. The less novel fanfictions generally receive better-than-average recognition, and vice versa. This result prompts us to perform more sophisticated analysis with proper control of confounding factors.

\begin{figure*}
    \centering
          \includegraphics[width=0.8\textwidth]{/scatter_all_log.pdf}
        \caption{The relationships between novelty and success, measured by kudos, hits, comments, and bookmarks. The horizontal axes are the novelty scores, and the vertical axes are the corresponding average of the z-score of kudos, hits, comments, and bookmarks in bins with bin size = 0.1 (left) and 0.05 (right). The confidence intervals obtained from bootstrap resampling are shown. }
        \label{fig:tfidf_lda_kudos}
\end{figure*}


\textbf{Regression analysis}   We perform several regression analysis to further explore this relationship. Four models are created with response variables being the logarithm of kudos, hits, comments, and bookmarks respectively. The term novelty and topic novelty scores are the predictor variables in all models. To account for possible non-linear relationships such as the inverse U-shape curve, we also use the square value of the term novelty and topic novelty scores as predictor variables. The values are mean-centered to avoid multicollinearity.

Control variables are added to account for the influence of other factors on the reception of fanfictions. First, the number of chapters that a fanfiction has is included, considering that multi-chapter fictions have more chances of exposure and that higher kudos may stimulate writing of more chapters. Secondly, categorical variables are created to capture the effects of the fandoms, archive warnings, categories, and age ratings. Thirdly, the authors may accumulate fame by writing fanfictions, and such fame may bias readership and kudos. We therefore include the total number of fictions that an author has written as a control variable. Fourthly, the relationship between characters is one of the main reasons that many fans read fanfictions, and some relationships have a larger fan base than others. To account for this effect, we create a binary variable \emph{frequent relationship} to capture whether a fanfiction features a relationship that is among the top 5 most frequent relationships in the fandom. Finally, we include the age of each fanfiction as a numerical variable, which is the number of days since a fanfiction was last edited. The correlations between the numerical variables are shown in Figure \ref{fig:corr}, showing no strong correlation between the predictor variables.

Because of the prevalent zero values in the response variables, we apply a Heckman--style two-step correction. A Logistic regression is first performed on the predictor and control variables to predict the probability of each sample having a non-zero outcome. This probability is then used as an additional predictor variable in a pooled OLS regression for each model.

\begin{figure}
    \centering
          \includegraphics[width=0.5\textwidth]{/variables_corr.pdf}
        \caption{Correlations between the numerical predictor, outcome, and control variables. Strong collinearity exists between the outcome variables, but not the predictor variables. }
        \label{fig:corr}
\end{figure}

\begin{figure*}
    \centering
          \includegraphics[width=0.8\textwidth]{/ols_coefs_partial.pdf}
        \caption{OLS coefficients for the independent variables and selected control variables. 95\% confidence intervals are shown. The coefficients of the categorical variables are omitted.}
        \label{fig:ols}
\end{figure*}

The OLS coefficient estimates of each model are shown in Figure \ref{fig:ols}. Note that a fanfiction featuring frequent relationships are likely to have better reception. In contrast to our assumption, the number of chapters, the author's fame, and the age of a fanfiction do not have significant influence on its success. While the coefficients of the categorical control variables are not shown, we have noted that fanfictions published in newer and more popular fandoms such as \emph{Star Wars} and \emph{Marvel}\footnote{These fandoms may have long histories, but recent installments such as Star Wars' new triology and the Marvel movies have brought in many new fans.} tend to be more successful. Elements such as character death and mature contents are associated with poorer recognition. 

After controlling for these effects, we find that the term novelty and its square value has significant negative coefficients close to zero, except for a positive coefficient for the squared of term novelty on the fictions' comments. This supports our previous observation that higher term novelty is linked to less kudos, hits, and bookmarks, while both very low and high term novelty are found to lead to more comments. Meanwhile, the topic novelty has significant negative coefficients in all models. The square value of topic novelty, however, has significant positive coefficients for comments and bookmarks. Both fictions with low and high topic novelty are therefore associated with more comments and bookmarks. 

To more closely examine this non-linear relationship, we turn to the generalized additive models (GAM) \cite{wood2006generalized}. The same set of predictor variables, control variables, and response variables are used, except that we no longer include the square value of term novelty and topic novelty. The estimated models are shown in Figure \ref{fig:gam}. In the case of term novelty, we observe a fluctuating decrease of kudos, hits, and bookmarks, but an increase in comments for high term novelty. This further supports that fanfictions with high term novelty are likely to receive more comments. For topic novelty, an overall monotonic decrease is found for all response variables, except the sharp rise in kudos and hits for high topic novelty. These two cases are found to be influenced by a few outliers, which have extremely high topic novelties and enjoy huge success (see Discussion). 

\begin{figure*}
    \centering
    \begin{subfigure}{\textwidth}
          \includegraphics[width=\textwidth]{/gam_results_term.pdf}
  		\caption{Term novelty} 
			\label{fig:gam_term}
       \end{subfigure}
      \begin{subfigure}{\textwidth}
          \includegraphics[width=\textwidth]{/gam_results_topic.pdf}
		\caption{Topic novelty} 
		\label{fig:gam_topic}
       \end{subfigure}
           	\caption{Results of the Generalized Additive Models. 95\% confidence intervals are plotted.}
        \label{fig:gam}
\end{figure*}



\section*{Discussion}
Traditional theories about pop culture suggest that people like things with a large degree of familiarity and a little surprise. Our findings with fanfictions diverge from this pattern: people in general do not like much surprise and tend to stay with the familiar. Occasionally, however, very surprising things are well-accepted. Our finding can explain many cultural phenomena: for example, people's desire to listening to old songs and the abundance of sequels and spin-off movies. Additionally, it also account for the emergence of high novelty works: sometimes, non-stereotypical creative works enjoy great succes.

Such cases can be found in our fanfiction dataset. In Figure \ref{fig:gam_topic}, the sharp rise in kudos and hits for high topic novelty can be attributed to a few fanfictions. The most extreme one is a fiction in the \emph{Marvel} fandom ``told from the perspective of Groot''. The fiction consist of the repetition of a single sentence ``I am groot.'' and received 67,219 kudos --- the highest number of kudos in AO3. It is a perfect example of a high novelty fanfiction receiving good recognition. Similarly, high novelty fictions like this may receive more comments, kudos and hits (see Figure \ref{fig:gam}). In many cases, readers do not actively seek for this type of fiction; but they welcome them when some highly innovative authors create them. This feedback process can often be observed in many contemporary arts, designs, etc.

The fanfiction dataset naturally controls for the variation in the fictions' subjects, characters and settings, allowing us to isolate the influence of novelty. We may therefore also avoid the confounding factors that may have contributed to the inverted U-shape curve that previous researches found. However, our dataset also has some limitations. Most writings in the dataset are produced by amateur writers, and do not have the same quality as published fictions. Additionally, the fanfictions are produced and consumed by a specific population, most of whom are young female fans of the contemporary pop culture. As transformative works, fanfictions are also ``secondary'' creative works that are different from most original creative works. 

We have developed a method to quantify the novelty of writings by comparing its linguistic features to other writings. We also measure novelty on two levels: the term level and the topic level. Our method, however, uses a bag-of-words model and neglects the semantic information contained in word orderings. It is also unable to disintegrate the novelty of style and the novelty of content, which may affect the readers' reception in different ways. Despite such caveats, our method provides an intuitive quantitive definition for novelty that can easily extend to other types of creative writings. A potential research direction is therefore to investigate the relationship between novelty and success in other areas, such as book markets or news articles.

\section{ Acknowledgments}
We thank the AO3 staff for helping us with the data collection, and thank Agnes Horvat for her comments.

\bibliography{main.bib}
\bibliographystyle{aaai}

\appendix
\section{Appendix}
\subsection{Model details}

The Python library \texttt{scikit-learn} was used to create the vector space models. 

The Python library gensim is used to fit LDA topic models on our data. Number of topics $=$ 100, $\alpha = 0.01$, iterations $=$ 50.




\end{document}



  In particular, the consumers' choices are largely shaped by the herding behavior. For example, it was found that being on the New York Times bestseller book list will cause an increase in sales \cite{sorensen2007bestseller}.  Even a ``faked'' popularity can be turned into real popularity as people are more likely to view creative works that they perceive to be popular \cite{salganik2008leading}. Therefore, such metrics may not accurately reflect the consumers' actual enjoyment of a cultural product. Moreover, as the features of cultural products vary significantly across genres, lengths, subjects, etc, it is hard to measure the novelty or conventionality of one piece against the others. 
  
  
 \begin{table*}
\centering
\begin{tabular}[width=0.8\textwidth]{p{4cm}p{3cm}p{3cm}p{3cm}p{3cm}}
\toprule
Response variable: & Kudos & Hits & Comments & Bookmarks \\ 
   \hline			
Term novelty & -0.6844*&  -0.4710*  & -0.2190* & -0.6298* \\
& (0.008) & ( 0.010) & (0.019) & (0.014)\\
Topic novelty & -2.3866* & -1.5971* & -0.9514* & -4.0380* \\
& (0.047) & (0.058) &  (0.026 ) & (0.077)\\
Chapters & -0.0075*& -0.0049* & -0.0002*& -0.0041* \\
& (0.000) & (0.000) & (8.01e-05) & ( 0.000) \\
Frequent relationship & 0.3765* & 0.4174* & 0.1881* & 0.6353*\\
& (0.006) & (0.007) & (0.013) & (0.010)\\
Author work count & -3.377e-05 * &-2.661e-05* & -5.21e-06* & -2.3e-05*\\
& (2.65e-06) & (3.28e-06) & (1.49e-06) & (4.33e-06)\\
Category (Female/Male) & -0.4615* & -0.2245* & -0.1142* & -0.3323*\\
& (0.006) & (0.007) & (0.009) & (0.009)\\
Category (Male/Male) & 0.1990* & 0.2637* & 0.1325* & 0.5224*\\
& (0.007) & (0.007) & (0.010) & (0.011)\\
Archive warnings (death) & -0.3719* & -0.2273* & -0.1432* & -0.4776*\\
& ( 0.010) & (0.012) & (0.010) & (0.017)\\
Rating (General audience) & -0.0339* & -0.1343* & -0.1163* & -0.6089*\\
& ( 0.008) & (0.010) & (0.007) & (0.015)\\
Rating (Mature) & -0.3559* & -0.4045* & -0.1171* & -0.4461*\\
& (0.008) & (0.009) & (0.009) & (0.012)\\
Fandom (Doctor Who) & -0.3892* & -0.3684* & -0.0969* & -0.5682*\\
& (0.006) & 0.008 & (0.007) & (0.010)\\
Fandom (Star Wars ) & 0.3575* & 0.2516* & 0.2331* & 0.2790*\\
& (0.016) & (0.020) & (0.016) & (0.026)\\
Age & 8.517e-05*& 0.0002* & -0.0001*& 1.296e-05*\\
& (4.53e-06) & (5.74e-06) & (9.09e-06 ) & (7.18e-06)\\
\hline 
Observations & 526992 & 526992 & 526992 & 526992 \\
Adjusted R-squared & 0.236 & 0.115 & 0.114  & 0.149\\
\bottomrule
\end{tabular}
 \begin{tablenotes}
      \small
      \item Standard errors are in parentheses.
      \item *p $<$ 0.001.
       \end{tablenotes}
\caption{OLS results}
\label{tab:regression}
\end{table*}%




\subsection*{Regression analysis}
We use the Python library statsmodels \cite{seabold2010statsmodels} to perform the pooled OLS regression. The GAM models are fitted using the pyGAM library \cite{pygam}, with parameters $splines=25$ and $lambda=300$.



Even when they seek out new fictions to read, they prefer to read things similar to what they have read before. This may be part of the nature of fan works --- because fans desire to see familiar characters and stories, it is reasonable for them to prefer fanfictions with familiar elements. The reasoning may extend to other parts of pop culture, 
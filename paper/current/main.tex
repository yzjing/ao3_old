\documentclass[11pt]{article} %{{{

\usepackage{amsmath}
\usepackage{amssymb}
\usepackage{graphicx}
\usepackage{url}
\usepackage[usenames,dvipsnames,svgnames,table]{xcolor}
\definecolor{light-gray}{gray}{0.8}
\def \del #1{ {\color{light-gray}{#1}} }
\def\yy#1{\footnote{\color{red}\textbf{#1 -YY}} }

\usepackage{biblatex}
\addbibresource{main.bib}

%}}}

\begin{document} %{{{

\title{Evolution of Writings in an Interest-oriented Online Community} %{{{
\date{\today}
\maketitle %}}}

\section{Introduction} %{{{
\label{sec:introduction}
\paragraph{} Many communities in the Internet are formed by a group of people sharing common interests. In particular, the contemporary pop culture sees the rise of a specific kind of online communities - fandoms. Fandoms, by convention, are a group of people that identify, connect and interact with each other based on a similar interest, such as a movie, a book or a band \cite{wiki:fandom}. 

\paragraph{}Transformative works, or in a more common term, fan works, are one typical kind of production from fandoms\cite{wiki:transf_work}. These are creative works made by fans based on one or more original works, and are often centered around certain characters or story lines. For example, a story written by a contemporary fan about Sherlock Holmes in his retirement is considered a fan fiction in the Sherlock Holmes fandom. Although fan works contain multiple media types such as art, music and games, one of the most common type is creative writing--fanfictions.

\paragraph{} This project studies the evolution of fanfictions in fandoms. Data for the study comes from the online transformative work archive site Archive of Our Own (AO3). This site allows free host for works that users upload, and categorizes them based on fandoms. It also utilizes a metadata system to store the works’ information, allowing for many filtering and classifying operations. Established in 2010, it has become one of the most popular transformative work archives.

\paragraph{} Besides the work texts, the metadata collected includes 23 fields that can be roughly divided into two categories. One is tags generated by the author, which describes the work’s content; the other is tags that are automatically generated, and describes the works’ other information, as well as the readers’ feedback. Table 1 gives the names of these fields. 
%}}}

\begin{table}[htdp]
\caption{Metadata of the writings}
\begin{center}
\begin{tabular}{p{7cm}|p{7cm}}
  \hline			
 Content related & Non-content related\\\hline
Additional Tags, Archive Warnings, Category, Characters, Fandoms, Rating, Relationship, Summary, Text, Title
&  Author, Bookmarks, Chapter Index, Chapters, Comments, Complete Date, Hits, Kudos, Language, Notes, Publish Date, Update Date, Word Count\\
\hline
\end{tabular}
\end{center}
\label{default}
\end{table}%


 



\section{Results} %{{{
\label{sec:results}

%}}}

\section{Methods} %{{{
\label{sec:methods}

%}}}
\nocite{*}

\printbibliography
    
\end{document} %}}}

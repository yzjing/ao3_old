%% PNAStwoS.tex
%% Sample file to use for PNAS articles prepared in LaTeX
%% For two column PNAS articles
%% Version1: Apr 15, 2008
%% Version2: Oct 04, 2013

%% BASIC CLASS FILE
\documentclass{pnastwo}

%% ADDITIONAL OPTIONAL STYLE FILES Font specification

%\usepackage{pnastwoF}



%% OPTIONAL MACRO DEFINITIONS
%\def\s{\sigma}
%%%%%%%%%%%%
%% For PNAS Only:
%\url{www.pnas.org/cgi/doi/10.1073/pnas.0709640104}
%\copyrightyear{2008}
%\issuedate{Issue Date}
%\volume{Volume}
%\issuenumber{Issue Number}
%\setcounter{page}{2687} %Set page number here if desired
%%%%%%%%%%%%

\begin{document}

\title{Consensus on Characterization with  in Fandom Communities}

\author{Yizhi Jing\affil{1}{Indiana University Bloomington},
}

\contributor{2015Fall I590 Final Paper}

\maketitle

\begin{article}
\begin{abstract}
{I present a study about the development ofagreement on characterization of fictional characters through textual analysis of transformative works. From analysis of tags that capture personalty traits, it was found that }
\end{abstract}

\keywords{text analysis | social systems}

Among the many types of dynamics in online communities, consensus of opinion is probably one of the most difficult ones to study. Depending on the specific nature of online communities, consensus of opinions may be attempted to reach on different topics: foreign policy, a product, a technology, or a film star. However, quantitative measurements of consensus are not easily conducted on most of this phenomena. [cite something]

Fandom communities are a group of fans that identify, and connect with each other based on a similar interest, such as a movie, a book or a band [cite https://en.wikipedia.org/wiki/Fandom]. A lot of fandom activities has been happening online, making them a specific type of online communities.

Transformative works, or fan works, are one typical kind of production from fandom communities. These are creative works made by fans based on one or more original works, and are often centered around certain characters or story lines. For example, a fiction written by a contemporary fan about Sherlock Holmes in his retirement is considered a fan fiction in the Sherlock Holmes fandom. By this nature, these works reflect the understanding of the creators about the characters and the story, which differs from person to person and changes over time.

I base my work on the assumption that analysis of transformative works may reveal the development of consensus of opinion between fans. In particular, this development may correspond to the typical life stages of a fandom, such as its origin, growth, decline and death. 

The data for this analysis comes from the online transformative work archive site Archive of Our Own (AO3)[cite: http://archiveofourown.org/]. This site allows users to freely upload their works, and categorizes them based on fandoms. It also utilizes a metadata system to store these works' information, allowing for many filtering and classifying operations. Established in 2010, it has become one of the most popular transformative work archives.

For this study, I used a python script to automatically collect data from this site. Besides the work texts, the metadata collected include 20 fields that can be roughly divided into two categories: some are generated by the author, which describes the work's content; others are automatically generated, and describes other work information, as well as the readers' feedback. Table 1 gives the names of these fields.

As a starting point, the project is also limited to a comparative study of two related fandoms: \textit{Sherlock(TV) }and \textit{ Sherlock Holmes - Arthur Conan Doyle }, being respectively the BBC TV series starring Benedict Cumberbatch and Martin Freeman, and the fiction by Sir Arthur Conal Doyle. \textit{Sherlock(TV) } is chosen as the case for study because of two reasons: first, it is among the fandoms with largest amounts of data, covering $~3\%$ of all works on the site with 77, 510 works. Secondly, AO3 opened on January 1st 2010, while the first episode of \textit{Sherlock} was broadcasted  o July 2010, which implies that the archive was able to record the origin and growth of the fandom. Meanwhile, the fandom for Conan Doyle's fiction has a longer history, but also a much smaller size.

The analysis shows x major findings. 






\section{Method}

\section{Results}

\section{Discussion}


\end{article}

\begin{table}
\centering
\caption{Fields of AO3 work metadata}
\begin{tabular*}{\hsize}{@{\extracolsep{\fill}}lcr}
Metadata type&Fields\cr
\hline
Author generated&author, title, summary, fandom, \cr
 &characters, relationship, category, rating, \cr
 &additional tags, archive warnings, notes\cr
 \hline
System generated&publish date, complete date, chapter count,\cr
 & word count, bookmarks count, chapters count,\cr
  & hits, kudos, language, comment counts \cr
\hline
\end{tabular*}
\end{table}

\end{document}



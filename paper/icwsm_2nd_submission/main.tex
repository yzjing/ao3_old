


\def\year{2018}\relax
%File: formatting-instruction.tex
\documentclass[letterpaper]{article} %DO NOT CHANGE THIS
\usepackage{aaai18}  %Required
\usepackage{times}  %Required
\usepackage{helvet}  %Required
\usepackage{courier}  %Required
\usepackage{url}  %Required
\usepackage{graphicx}  %Required
\frenchspacing  %Required
\setlength{\pdfpagewidth}{8.5in}  %Required
\setlength{\pdfpageheight}{11in}  %Required

\usepackage{subcaption}
\usepackage{booktabs}
\usepackage{mathtools}
\usepackage[flushleft]{threeparttable}

% temporary formatting for editing
%\linespread{2}
% \pagestyle{plain} 


\graphicspath{ {./figs} }

\DeclarePairedDelimiterX{\infdivx}[2]{(}{)}{%
  #1\;\delimsize\|\;#2%
}
\newcommand{\infdiv}{D\infdivx}
\DeclarePairedDelimiter{\norm}{\lVert}{\rVert}

% Novelty Disturbs, Sameness Attracts, and Outliers Flourish in Fanfiction
% Sameness Attract, Novelty Disturbs, but Outliers Flourish in Fanfiction Online
%PDF Info Is Required:
  \pdfinfo{
/Title (Sameness Attracts, Novelty Disturbs, but Outliers Flourish in Fanfiction Online)
 /Author (Elise Jing, Simon DeDeo, Yong-Yeol Ahn)
}
\setcounter{secnumdepth}{0}  
\begin{document}
% The file aaai.sty is the style file for AAAI Press 
% proceedings, working notes, and technical reports.
% Resist the Unusual: 
% Does Novelty Disturb and Repel? Linguistic features and the Success of Fanfictions in Fandoms)
% Fanfiction likes extremes: sameness attracts, novelty disturbs, and outliers flourish i
% resist the unusual: Novelty Disturbs, Sameness Attracts in Fanfiction
\title{Sameness Attracts, Novelty Disturbs, but Outliers Flourish in Fanfiction Online}
\author{Elise Jing, Simon DeDeo, Yong-Yeol Ahn}

%% Simon address/e-mail: sdedeo@andrew.cmu.edu, Social and Decision Sciences, Dietrich College, Carnegie Mellon University, 5000 Forbes Avenue, Pittsburgh, PA 15015 and Santa Fe Institute, 1399 Hyde Park Road, Santa Fe, NM 87501 
\maketitle
\begin{abstract}
The nature of what people enjoy is not just a central question for the creative industry, it is a driving force of cultural evolution. It is widely believed that successful cultural products balance novelty and conventionality: they provide something familiar but at least somewhat divergent from what has come before, and occupy a satisfying middle ground between ``more of the same'' and ``too strange''. We test this belief using a large dataset of over half a million works of fanfiction from the website Archive of Our Own (AO3), looking at how the recognition a work receives varies with its novelty. We quantify the novelty through a term-based language model, and a topic model, in the context of existing works within the same fandom. Contrary to the balance theory, we find that the lowest-novelty are the most popular and that popularity declines monotonically with novelty. A few exceptions can be found: extremely popular works that are among the  highest novelty within the fandom. Taken together, our findings not only challenge the traditional theory of the hedonic value of novelty, they invert it: people prefer the least novel things, are repelled by the middle ground, and have an occasional preference for extreme outliers. It suggests that cultural evolution must work against inertia --- the appetite people have to continually reconsume the familiar, and may resemble a punctuated equilibrium rather than a smooth evolution.
\end{abstract}


\section*{Introduction}
\noindent 
A central puzzle for the arts and cultural industries, a sector that contributes more than \$700 billion to the U.S. GDP in 2018 by products such as movies, TV shows, and video games \cite{artsculturalindustries}, is to predict what people will like and enjoy. As suggested by the very name ``creative work'', people may seek surprise when consuming artistic and cultural creations \cite{hutter2011infinite}. At the same time, we know that people have a strong preference for familiarity. When people listen to music, for example, they often choose songs that they are familiar with \cite{thompson2014shazam}. Such preferences may be basic:  humans also find high attractiveness in faces that are close to the average, and therefore more familiar, and the effect even extends to images of other objects such as birds and automobiles \cite{Halberstadt2003}. Moreover, the mere exposure effect found by Zajonc show that exposure to a certain stimuli can increase people's preference for it \cite{zajonc1968attitudinal}, even when they are not aware of the exposure \cite{kunst1980affective,bornstein1989exposure}. 

``Looking like what has come before'', leveraging the familiar, is therefore expected to be a core strategy to achieve popularity. Repetition has long been a central feature of cultural products such as music and poetry \cite{huron2013psychological}, and  the trend persists in contemporary popular culture, where new releases are often adaptations, remakes, and remixes of existing works \cite{manovich2007comes}. Marvel and DC Comics have been making massively successful movies based on their popular comics, including many sequels, prequels, reboots, and soft reboots. The origin story of Spider-man provides an extreme case: it has been re-played in three movies since 2002 --- \emph{Spider-Man} (2002), \emph{The Amazing Spider-Man} (2012) and \emph{Spider-Man: Homecoming} (2017) \cite{spiderman}. Market reception indicates that the audience enjoy such repetitions: the three movies grossed about \$640 million, \$300 million, and \$340 million respectively (inflation-adjusted) \cite{spider-gross}, among the top 300 highest-grossing movies of all time. Out of the 10 highest-grossing movies of 2016, 8 are adaptions, sequels, remakes, or parts of a movie universe \cite{2016film}. This number increased to 10 out of 10 in 2017 \cite{2017film}. Notably, many such films also follow the typical plots of their genres. Film and literary theories have further suggested that most stories can be summarized into a few archetypes and basic elements, such as the Hero's Journey \cite{campbell2008hero}, mythemes \cite{levi1955structural} or Proppian functions \cite{propp2010morphology}. In these accounts, every story involves the repetition of classical narrative elements. The preference for repetition can be found in political discourse just as well as cultural artifacts: for instance, in Parliament of France during the Revolution, the more novel a speech is, the less likely its combinations are to persist in later speeches \cite{barron2018individuals}.

Extremely novel cultural products, on the other hand, sometimes also enjoy huge successes. In the film industry, such movies are often landmark introductions of new forms of cinematography, such as \emph{Star Wars} (multiple ground-breaking technologies) and \emph{Avatar} (the first mainstream 3D film using stereoscopic filmmaking). Popular music in the 20th Century is characterized by high-turnover and the emergence of new genres \cite{mauch2015evolution}, such as hip-hop and electronic dance music, that start as underground music and growing to enjoy global commercial success. In the history of literature, experimental, deviating works such as \emph{The Trial} by Franz Kafka and \emph{Ulysses} by James Joyce are considered canonical events that define both the author and the era. One of the most influential figures in the French Revolution, Maximilien Robespierre, reliably delivered some of the most novel speeches. If people only enjoy familiar things, what accounts for such success?

\textbf{``Balance Theories'' of Liking}. A widely-adopted hypothesis reconciles the apparent conflict between the preference for novelty and familiarity by suggesting that successful creative works are a combination of, or balance between, convention and innovation. Under this theory, popular works are different from previous works and their peers, but not \emph{too} different. In psychology, this idea is captured in the Wundt-Berlyne curve, an inverted U-shaped curve with an optimal amount of novelty for hedonic values \cite{berlyne1970novelty}; a similar account is found in the business and marketing literature where it is known as Mandler's Hypothesis \cite{meyers1989schema}. 

A few experiments have supported this hypothesis in the case of words \cite{sluckin1980liking}, music \cite{hargreaves1984effects,askin2017makes},  films \cite{sreenivasan2013quantitative}, advertising \cite{mohanty2016visual}, and scientific publications \cite{uzzi2013atypical}, suggesting that, for example, songs with an optimal amount of differentiation are more likely to be on the top of the Billboard's Hot 100 charts, that movies balancing familiarity and novelty have the highest revenues, or that visual metaphors in ads work best when they have mild incongruity. In scientific publications, the highest-cited papers are grounded on mostly conventional, but partly novel combinations of previous works. However, much of these research deals with cultural products whose receptions are influenced by many factors: not just intrinsic factors such as content, style, subject, genre, or length, but also external ones such as price, advertisement, and media coverage. For example, a song's popularity is only partially determined by its quality \cite{salganik2006experimental}, and deliberately putting unpopular songs on the top chart can in fact popularize them \cite{salganik2008leading}. The complex interactions between consumers, cultural products and the market is therefore difficult to disentangle. 

In order to verify the balance theory while controlling for such interactions, we study a special dataset that allows isolating the effect of novelty, which is a dataset of fanfictions (see Data and Methods). Using this data, we investigate how a fanfiction's novelty is associated with its success. We use two frameworks to characterize the novelty of the fiction: a term based language model and topic modeling. In both cases, a fiction is evaluated with respect to the existing fictions in the same fandom. A fiction is more novel if it is more different from the other fictions published during the same time period in the feature space. The success of a fiction is evaluated by its readers’ responses, such as liking and commenting (see Data and Methods). 


Our statistical methods recover a monotonically decreasing relationship between a fiction's novelty and its success with, in some exceptional cases, an additional uptick for very high novelty work. Readers prefer creative fictions that give them a sense of familiarity, and we find no limit to their appetite for ``more of the same''; the exception is provided by extreme outliers. Taken together, the main contribution of our findings is that they reverse the usual ``inverted  U-shape'' account, and diverge from the balancing theories. Instead of a sweet spot at the intermediate level of novelty, we found a lack of recognition for intermediate values of novelty. Success does not come from combining novelty and sameness, but from pushing one of these to the extreme.


\section*{Data \& Methods}

 
\textbf{Fanfiction and Fandoms}. A special type of creative work, known as fan work, allows us to avoid some of the confounds in studying creative works. Fan works draw on narratives and characters from a particular canonical work to create new stories and alternate timelines \cite{wiki:transf_work}. While fan works appear in multiple media including painting, music and games, the most common type is creative writing, or ``fanfiction''. An enthusiast of the Harry Potter series may write a new adventure for Hermione and Harry after the end of Rowling's novels; a fan of the television show Buffy the Vampire Slayer may write a story in which Willow Rosenberg's girlfriend has a different fate. 

Fanfiction is often transformative and transgressive: fans of Sherlock Holmes, for example, have written stories in which Holmes and Dr.\ Watson fall in love, and Watson, by magical means, gestates a baby they conceive together. Examples like these abound: despite being based around a canonical work, fanfiction communities are far from conservative or uncreative and often, to the contrary, attempt not only to extend a canonical work, but also to subvert it. Fanfiction, in short, can be seen not as an outlier of cultural production, but as a paradigm for textural production in general, characterizing the ways in which one author or story influences another~\cite{rohrs6north}, and  unusual only in how explicitly it identifies the ancestor from which downstream texts evolve. Fanfiction has drawn attention from media studies and cultural studies (see, \emph{e.g.}, \cite{thomas2011fanfiction}). However, these works have often focused on the social aspect of fanfiction: the identity of the writers \cite{black2006language}, the practice of writing \cite{LIT:LIT12061}, and the interaction between fans \cite{hills2015expertise}, rather than on the texts themselves as creative works.


\begin{table}
\centering
\begin{tabular}[width=0.9\textwidth]{p{3cm}p{5cm}}
\toprule
Fields & Description \\ 
   \hline

Title & Title of the work. \\
Fandoms & The fandom(s) the work belongs to. \\
Author & The author of the work.  \\
Chapters & The number of chapters that the work has. \\
Archive Warnings & Warnings for sensitive elements. \\
Category & The type of relationships in the work. \\
Rating & The age rating. \\
Relationship & The relationship(s) between characters in the work, in the form of Character A/Character B
or Character A\&Character B. \\
Publish Date & The date the work was published on AO3. \\
Complete date & For multi-chapter works, the date when it was marked as ``complete''.\\
Update date & The date when the work was last updated. \\
\hline

Kudos & The number of kudos (likes) the work received. \\
Comments & The number of comments the work received.\\
Hits & The number of times the link to the work was clicked on. \\
Bookmarks & The number of people who bookmarked the work.\\

\bottomrule
\end{tabular}
\caption{Metadata fields for each work of fanfiction in our database.}
\label{tab:metadata}
\end{table}%


\begin{figure*}
    \centering
   % \hspace{1cm}
        \includegraphics[width=\textwidth]{/fic_time_fandom_size_dist.pdf}
        \caption{The size of fandoms and the amount of fanfiction published in AO3 each month, from 2009 to 2016.}
        \label{fig:fandom_size}    
    \end{figure*}
    
The nature of fanfiction allows us to avoid a number of common confounds in the study of creativity and what people enjoy. First, fanfiction is usually shared among relatively isolated online communities (``fandoms'') with almost no advertisement or promotions \cite{wiki:fandom}; a work's success is therefore largely uninfluenced by top-down interventions, such as advertising campaigns, that can distort its reception. Second, most fanfiction works are freely available in the Internet, so that the price is not a confounding factor. Thirdly, works in the same fandom are created within the same context, and have similar subjects, characters, and settings to each other. As in genres of literary production, variations between works occur in a recognizable space, allowing us to talk about their novelty while controlling for other factors. Finally, because fanfiction is freely available in plain text and in remarkably large volumes, natural language processing methods can be used to operationalize novelty for quantitative studies. 



We draw our data from the online fanfiction archive Archive of Our Own (AO3), which allows users to upload their work, and categorizes them based on fandoms. Established in 2009, it has become one of the largest fan communities, with more than 1.6 million users and 4 million works by August 2018 \cite{ao3stats}. A Python script was used to download fanfictions and their metadata from AO3 (\url{http://archiveofourown.org/}) in March 2016. 

    
\begin{figure*}
    \centering
       \includegraphics[width=\textwidth]{/kudos_comments_hits_bookmarks_ccdf.pdf}
        \caption{Log-binned probability density function and complementary cumulative distribution of kudos, hits, bookmarks and comments. For multi-chapter fanfictions, we average these values over the number of chapters. Fat-tailed distributions are observed, where a small portion of fictions receive many kudos and comments, and most receive few.}
        \label{fig:kudos_dist}
    \end{figure*}

AO3 classifies the fandoms based on the formats of thecanons, such as movies, TV shows, books, anime \& manga,and musicals. We first identify the top 5 fandoms in each of these categories based on the number of works they contain. We exclude the fandoms that are subsets of other large fandoms. For example, we keep \emph{Marvel} but exclude \emph{The Avengers}, because \emph{The Avengers} is a part of the Marvel Universe. Fandoms that do not have a unified subject, such as K-pop (which contains fanfictions about over 300 different Korean pop bands), were also removed. We keep only the fictions written in English. Finally, to control for the effects of the work's length on both reader preference and on our methods, we only analyze the fictions with 500--1500 words in each chapter. This leaves us with 666,430 works in 23 fandoms from 90,065 authors.

Figure \ref{fig:fandom_size}(a) shows the number of works in each of the 23 fandoms; \emph{Marvel}, \emph{Supernatural}, and \emph{Sherlock Holmes} are the largest. Figure \ref{fig:fandom_size}(b) shows the volume of works produced over time. AO3 was established in December 2009, and experienced rapid growth beginning in 2012. Because works timestamped earlier than the start date might be migrated from other platforms, and may not correctly reflect the status of the archive, we only run our analysis using the fictions published in January 2010 or later.

Metadata for each fanfiction (see Table \ref{tab:metadata}) allows us to gain insights about the fictions and their reception. The variables indicating kudos (a ``like''), hits, comments, and bookmarks are used to operationalize the success of a work. While the hits is the most direct metric for popularity, it may be strongly affected by the title, summary, and the author, rather than the content itself. The kudos is a clearer signal that a reader liked the fiction. A reader can also comment on a fiction or bookmark it to read later. The comments and bookmarks therefore signal the recognition or engagement from readers, although they depend on multiple motivations, and are less directly associated with popularity. Figure \ref{fig:kudos_dist} shows the distribution of these indicators on a logarithmic scale. Like many other measurements of popularity, they exhibit fat tails, with a small number of works receiving the majority of attention and most receiving little or none. Also note that there are outliers that achieved extreme recognition in terms of hits and kudos. In the following analysis, we log-transform these values, a common practice for similar data such as citations \cite{thelwall2014regression}. This practice has been argued to reduce the potential bias when performing regression and other statistical analysis \cite{thelwall2014regression}. 

\textbf{Quantifying novelty}. Although there are many ways to measure novelty, here we operationalize it in an intuitive, data-driven way. As previous studies noted \cite{askin2017makes,de2015game}, it is reasonable to assess a work's novelty in the context of other previously published works. Intuitively, a work is less novel if it is similar to many others published beforehand. Here we use the centroid, in feature space, of all past works in a fandom as the guide for measuring novelty. A work is more novel the further it is from the center. 

In line with many existing researches such as \cite{klingenstein2014civilizing,barron2018individuals,horvat2018role}, we extract features from the works using two methods, the term frequency--inverse document frequency (TF--IDF) and the Latent Dirichlet Allocation (LDA) \cite{blei2003latent} -- two  fundamental and widely used ways to model text data. TF--IDF is a vector space model often used in information retrieval \cite{salton1988term}, where each document is represented as a vector, and its entries are the TF--IDF scores of the unique terms in the document. It discounts the importance of commmon terms that appear in many documents. This allows us to quantify the term-level novelty. We first preprocess the texts by removing the terms with $frequency = 1$.The TF-IDF scores are then computed, for each fanfiction, using all fanfictions published in the same fandom within the past 6 months from when it was published. The Python library \texttt{scikit-learn} was used to create the vectors. We then compute the centroid of the feature space as the average of all feature vectors. The \emph{term novelty} score $s_{i}^{(\mathrm{term})}$ of a fanfiction is defined as:

\begin{equation}
s_{i}^{(\mathrm{term})} = 1-\frac{\boldsymbol{f_i} \cdot{\boldsymbol{f_i^{(c)}}}} {\lvert \boldsymbol{f_i} \rvert \lvert \boldsymbol{f_i^{(c)}} \rvert}
\end{equation}

where $\boldsymbol{f_i}$ is the vector representation of a fanfiction, and $\boldsymbol{f_i^{(c)}}$ is the centroid of the vector space defined with respect of this fanfiction.

LDA provides our second measure of novelty, characterizing documents in terms of topics, or co-occuring word patterns. Similar to the TF--IDF, for each work, we construct a feature space consisting of the topic distributions of the works published before it, and define the \emph{topic novelty} score $s_{i}^{(\mathrm{topic})}$ as the Jensen-Shannon Distance (JSD; \cite{klingenstein2014civilizing}) between the fanfiction's topic distribution and the center of the feature space:

\begin{equation}
\begin{centering}
s_{i}^{(\mathrm{topic})} = \frac{1}{2}D(\boldsymbol{f_i^{(c)}}\rvert\rvert\boldsymbol{v}) + D(\boldsymbol{f_i}\rvert\rvert\boldsymbol{v})
\end{centering}
\end{equation}

where $\boldsymbol{f_i}$ is the vector representation of a fanfiction, and $\boldsymbol{f_i^{(c)}}$ is the centroid (see above). $\boldsymbol{v} = \frac{1}{2}(\boldsymbol{f_i} + \boldsymbol{f_i^{(c)}})$, and D is the  Kullback–Leibler divergence between two vectors.

The Python library \texttt{gensim} is used to fit LDA topic models on our data. The parameters are: number of topics $=$ 100\footnote{It was found that varying the number of topics does not influence our results significantly.}, $\alpha = 0.01$, and iterations $=$ 50. The texts are preprocessed by removing the top 500 most frequent words and the words with $frequency = 1$.

The data and code that we used will be made available.

\section{Results}
First of all, we would like to validate our quantification of novelty by checking if the fictions found to be novel by our methods are also novel to human readers. We manually examined several high-novelty fanfictions and confirmed that they are indeed innovative in terms of topics and styles. One example is a fiction in the \emph{Marvel} fandom titled \emph{I am groot}. It has a term novelty score of 0.99 and is  ``told from the perspective of Groot''. The fiction consists of 437 repetitions of a single sentence ``I am groot.'' Similarly, one of the works with a novelty score of 0.99 in the \emph{Star Wars} fandom is about the exchange between C-3PO and R2D2, written in binary numbers. A high-novelty fiction in the \emph{Sherlock} fandom, titled \emph{The Real Meaning of Idioms}, is written in the form of text messages.  Another fiction with high topic novelty is found in the \emph{Doctor Who} fandom. Titled \emph{The Boy Who Waited}, it is a cross-over with the \emph{Marvel} Universe, in which characters from both fandoms coexist and interact. Such examples, albeit anecdotal by nature, show that the fanfictions found to be novel by our methods are quite novel to human readers.


 

\textbf{Correlation between novelty and success}. Figure \ref{fig:tfidf_lda_kudos} displays the raw correlation between a work's novelty and the kudos, hits, comments, and bookmarks it receives, aggregated across fandoms. Since the fandoms differ in size and activity, we compute the $z$-score for each value based on the average of its fandom, and use the logarithm values because of the aforementioned reasons. We observe that as the term novelty and topic novelty score increases, the $z$-score of kudos, hits, comments, and bookmarks decrease across the board, seemingly negatively correlated. In other words, novelty is associated with poor recognition and engagement, not concurring with the balance theory. This result prompts us to perform a more sophisticated multiple regression that controls for confounding factors.


 \begin{figure}[h]
    \centering
    % \vspace{-2cm}
          \includegraphics[width=0.5\textwidth]{/scatter_all_log.pdf}
        \caption{The relationships between novelty and success, measured by kudos, hits, comments, and bookmarks. The horizontal axes are the novelty scores, and the vertical axes are the corresponding average of the z-score of kudos, hits, comments, and bookmarks in bins with bin size = 0.1 (upper) and 0.05 (lower). The confidence intervals obtained from bootstrap resampling are shown. }
        \label{fig:tfidf_lda_kudos}
\end{figure}

\subsection{Regression}


\begin{table*}
\centering
\begin{tabular}[width=\textwidth]{p{2cm}p{5cm}p{5cm}p{3cm}}
\toprule
Model & Response variables & Independent variables & Control variables \\ 
   \hline
Models 1-4 & Logarithm of Kudos, hits, comments, and bookmarks respectively & Term novelty, topic novelty & All control variables \\
Models 5-8 & Logarithm of Kudos, hits, comments, and bookmarks respectively & Term novelty, square of term novelty, topic novelty, square of topic novelty & All control variables \\
Models 9-12 & Kudos, hits, comments, and bookmarks respectively & Term novelty, topic novelty & All control variables \\

\bottomrule
\end{tabular}
\caption{The response, independent, and control variables used in each of the regression models.}
\label{tab:reg}
\end{table*}%

\textbf{Response variables}.  Four response variables --- kudos, hits, comments, and bookmarks --- are considered. We use the logarithm values because these variables exhibit fat-failed distributions (see Data & Methods).


\textbf{Independent variables}. The term novelty and topic novelty scores are the predictor variables in all models. To account for possible non-linear relationships such as the inverted U-shape curve, we also use the square values of these scores as predictor variables. We first mean-center the term novelty and topic novelty scores before computing the square values.

\begin{figure}
    \centering
          \includegraphics[width=0.5\textwidth]{/variables_corr.pdf}
        \caption{Correlations between the numerical predictor, response, and control variables. Strong correlation exists between the response variables, but not the predictor variables. }
        \label{fig:corr}
\end{figure}

\begin{figure*}[h]
    \centering
    \begin{subfigure}{\textwidth}
            \includegraphics[width=\textwidth]{fanfic/figs/ols_coefs_partial_twoparts_nosq.pdf}
  		\caption{Models 1-4} 
			\label{fig:ols_partial_nosq}
       \end{subfigure}
      \begin{subfigure}{\textwidth}
          \includegraphics[width=\textwidth]{fanfic/figs/ols_coefs_partial_twoparts.pdf}
		\caption{Models 5-8} 
		\label{fig:ols_partial}
       \end{subfigure}
           	\caption{OLS coefficients for the independent variables and selected control variables for the multiple regression models. 95\% confidence intervals are shown. The coefficients of the categorical variables are omitted.}
        \label{fig:ols}
\end{figure*}


\textbf{Control variables}.   To isolate the relationship between the novelty and success, we consider the following control variables: (1) Some fandoms have larger fan bases than others, resulting in higher numbers of kudos in general for the works in these fandoms. We use binary variables that represent each fandom to control for this effect. (2) The number of chapters provides a second control: multi-chapter works have additional chances for exposure, and that higher kudos may stimulate an author to write more chapters. (3) Since AO3's user base has been increasing, one may expect newer works to receive more kudos than older ones. Conversely, older works may receive more kudos because they had more time to be discovered or to accumulate readers. We therefore include the age of each work as a numerical variable: the number of days since a work was completed (for finished works) or was last edited (for incomplete works). (4) Since the authors may accumulate fame through writing fanfiction, and this fame may bias readership and kudos,  we control for the total number of works that an author has written. (5) The relationship between characters is one of the main reasons that many fans read fanfictions, and some relationships have larger fan bases than others. To account for this effect, we create a binary variable, ``frequent relationship'', to indicate whether a work features a relationship that is among the top five most frequent relationships in its fandom. Finally, (6) the ``Archive Warnings'' indicate that the fanfictions contain sensitive elements such as graphic violence or major character death, and may influence the readers' choice to read them; the age ratings restrict some fictions to adults only; the types of character relationships may also influence the readers' choices. Categorical variables are created to capture their effects. The correlations between the numerical variables are shown in Figure \ref{fig:corr}, showing no strong pairwise correlation within the predictor and the numerical control variables. We also examined the variation inflation factor (VIF) and removed one variable that causes strong collinearity (the age of fictions), although keeping it does not strongly influence the results. The variables that each model contains are summarized in Table \ref{tab:reg}.

Because of the prevalent zero values in the response variables, we choose to use a two--part model \cite{jones2000health,humphreys2013dealing}\footnote{Because our outcomes are the \emph{average} of counts data, zero--inflated Poisson or negative binomial regression models are not appropriate.}. A logistic regression is first performed on the predictor and control variables to predict the probability of each sample having a non-zero outcome. This probability is then used as an additional predictor variable in a pooled OLS regression on the samples \emph{with non-zero outcome}. We use the Python library \texttt{statsmodels} \cite{seabold2010statsmodels} to perform both types of regression.

Selected OLS coefficient estimates of the models are shown in Figure \ref{fig:ols}. Let us first examine the coefficients of the control variables. A work featuring frequent relationships is likely to have better reception. In contrast to our assumption, the number of chapters and the author's fame do not have significant influence on its success. While the coefficients of the categorical control variables are not shown here, we found that work published in newer and more popular fandoms such as \emph{Star Wars} and \emph{Marvel}\footnote{These fandoms may have long histories, but recent installments such as Star Wars' new trilogy and the Marvel movies are associated with the influx of new fans.} tend to be more successful. Elements such as character death and mature contents are associated with poorer recognition. For full results with all coefficients, see Appendix.

\begin{figure*}
\vspace{-1cm}
    \centering
    \begin{subfigure}{\textwidth}
          \includegraphics[width=\textwidth]{/gam_results_term_20_300.pdf}
  		\caption{Term novelty} 
			\label{fig:gam_term}
       \end{subfigure}
      \begin{subfigure}{\textwidth}
          \includegraphics[width=\textwidth]{/gam_results_topic_20_300.pdf}
		\caption{Topic novelty} 
		\label{fig:gam_topic}
       \end{subfigure}
           	\caption{Models 9-12: results of the Generalized Additive Models. The x-axis: the term/novelty scores. The y-axis: the expected value of Kudos/Hits/Comments/Bookmarks using the novelty score as the predictor variable while holding other independent variables constant. 95\% confidence intervals are shown.}
        \label{fig:gam}
\end{figure*}


We then examine models 1-4, which do not include the square values of the term and novelty scores (Figure \ref{fig:ols_partial_nosq}). After controlling for these effects, we find that the term novelty and topic novelty have significant negative effects in all four models, supporting our previous observation that higher term novelty is linked to less success. For example, the topic novelty score increase by 0.05 (cf. Figure \ref{fig:tfidf_lda_kudos}) is associated with the decrease in kudos by approximately 37.8\% and the decrease in comments by 55.1\%, holding other variables constant. 

When we add the square values of novelty scores in models 5-8 (Figure \ref{fig:ols_partial}), the coefficients for the term and novelty scores are similar to that in models 1-4, supporting the robustness of our models. At the same time, we find that the coefficients of the topic novelty squared are larger than those of term novelty squared and positive across all cases. Both fictions with low and high topic novelty are therefore associated with better reception, suggesting not the \emph{inverted} U-shaped curves, but \emph{U-shaped} curves.

\textbf{GAM.} The linear regression model suggests a potentially nonlinear relationship between novelty and success, but cannot directly reveal details of such nonlinear relationship. We therefore turn to the generalized additive models (GAM)~\cite{wood2006generalized}, which allows for studying non-linear relationships in complex cultural data (e.g.~\cite{horvat2018role}). Here we use the raw values of kudos, hits, comments, and bookmarks as the response variables, instead of their logarithm, and use the same set of predictor variables and control variables as in models 1-4 (see Table \ref{tab:reg}).
The models are fitted using the \texttt{pyGAM} library, with parameters: number of splines $=$ 20 and $\lambda = 300$. The R library \texttt{mgcv} is used to tune the parameters. 

The estimated models are shown in Figure \ref{fig:gam}. In the case of term novelty, we observe a fluctuating decrease in the expected value of kudos, hits, and bookmarks, but an increase in comments for high term novelty. This further supports that works with extreme high novelty may achieve high engagement. For topic novelty, an overall monotonic decrease is found for all response variables, except the sharp rise in kudos and hits for the extreme topic novelty. These two cases are found to be influenced by a few outliers, which have extremely high topic novelties and enjoy huge success as we discuss shortly. 


\section*{Discussion}
Traditional theories of culture suggest that people like things with a balance of familiarity and surprise. Our findings from the fanfiction community contradict this: people, in general, are repelled by  novelty and tend to stay with the familiar. On rare occasions, however, extreme novelties gain huge attention and success.

These extreme cases can be seen in our results. 
In Figure \ref{fig:gam_topic}, the sharp rise in kudos and hits for high topic novelty can be attributed to a small number of fanfictions, such as the \emph{I am Groot} and the binary numbers examples discussed above (see Results). These fictions are all among the most well-received ones in their fandoms. The \emph{Groot} fiction in particular received 67,219 kudos --- the highest number of kudos ever recorded in AO3. These examples reflect some well-recognized ways of innovation: stylistic innovation, as well as recombination and remixing. The recognition of such highly novel creations may be one of the incentives for authors to risk the dangers of innovation (see Figure \ref{fig:gam}). Readers may not actively seek for this type of fiction, but they appreciate them when found. In the case of art and design, a ``surprised by novelty'' dynamic is perhaps best captured by the Steve Jobs quote: ``People don't know what they want until you show it to them.'' This dynamics further implicates that cultural evolution may not be a smooth process. Similar to the punctuated equilibrium theory in evolutionary biology \cite{gould1972punctuated} and the paradigm shifts in scientific revolution \cite{kuhn2012structure}, people have the inertia to keep consuming familiar things for their comfort, until some revolutionary creation redefine their tastes and open up a new space for followers, becoming the landmarks in cultural history.

The boundaries that fandoms impose on themselves provide a natural control for the variation in subjects, characters and settings, allowing us to better isolate the influence of novelty, and avoiding the confounding factors that may have contributed to the inverted U-shape curve found by previous studies. However, our dataset also has some limitations. Both the authors and the consumers of the works are drawn from a particular population skewed towards the young females, and largely from the United States and United Kingdom. Fanfiction is usually the domain of amateur writers whose training, socialization, and incentives may differ from the ``professional'' producers of culture in other domains. Finally, while all cultural practices have a chain of inheritance, fanfictions are more explicitly anchored in original texts than most.

Our results appear robust to two different characterizations of the texts (term, and topic); both methods, however, neglect the semantic information contained in word orderings. In literary theories, the arrangement of events plays an essential role in stories. Our methods capture the `material’ of stories but are unable to evaluate the way it is arranged. Moreover, the novelty of a piece of writing may appear in its style as well as in content. Some stylometric features, such as the usage of certain words, are captured by our methods, but they could not be decomposed from contents. Other features such as sentence length and punctuation usage are neglected by our methods. This treatment of stylometric features may bias our evaluation of novelty.

Our results may also diverge from previous researches because of the methods we used. In the early experiments by Berlyne \cite{berlyne1970novelty}, they controlled the novelty of geometric shapes by having the subjects exposed to them, and then evaluate success by asking the subjects to report their ``interestingness’’. Similarly, Zajonc’s experiments exposed subjects to groups of words \cite{zajonc1968attitudinal}. These experiments have very different data and setup from ours, which can be one reason for the different outcomes. However, we also note that our results diverge from some recent researches that measure novelty and success similar to our study \cite{horvat2018role,askin2017makes}, suggesting that our findings are not merely caused by the different experimental setups.


Despite the above caveats, we argue that our results are robust and have practical values. For fanfiction writers, our findings suggest they may gain popularity by writing fiction fitting into the ``mainstream’’ of the fandom. However, exceptional recognition sometimes come from creating an avant-garde, adventurous, and ingenious work. This recommendation may also apply to the broader area of genre fiction, where sameness may continue to satisfy readers, while high novelty can open up new markets. Our methods can also easily extend to other types of text data. A potential research direction is therefore to investigate the relationship between novelty and success in other areas, such as book markets or news articles.

% \section{Acknowledgments}
% We thank the AO3 staff for helping us with the data collection, and thank \' Agnes Horv\' at and Jaehyuk Park for their comments.

\bibliography{main.bib}
\bibliographystyle{aaai}

\appendix

\subsection{Appendix: Full multiple regression results}
The complete figure of coefficient estimates for models 5-8 are shown in Figure \ref{fig:ols_full}.

\begin{figure*}
    \centering
          \includegraphics[width=\textwidth]{fanfic/figs/ols_coefs_full_twoparts.pdf}
        \caption{OLS coefficients for models 5-8. 95\% confidence intervals are shown. N = 662793.}
        \label{fig:ols_full}
\end{figure*}



\end{document}



  In particular, the consumers' choices are largely shaped by the herding behavior. For example, it was found that being on the New York Times bestseller book list will cause an increase in sales \cite{sorensen2007bestseller}.  Even a ``faked'' popularity can be turned into real popularity as people are more likely to view creative works that they perceive to be popular \cite{salganik2008leading}. Therefore, such metrics may not accurately reflect the consumers' actual enjoyment of a cultural product. Moreover, as the features of cultural products vary significantly across genres, lengths, subjects, etc, it is hard to measure the novelty or conventionality of one piece against the others. 
  
  
 \begin{table*}
\centering
\begin{tabular}[width=0.8\textwidth]{p{4cm}p{3cm}p{3cm}p{3cm}p{3cm}}
\toprule
Response variable: & Kudos & Hits & Comments & Bookmarks \\ 
   \hline			
Term novelty & -0.6844*&  -0.4710*  & -0.2190* & -0.6298* \\
& (0.008) & ( 0.010) & (0.019) & (0.014)\\
Topic novelty & -2.3866* & -1.5971* & -0.9514* & -4.0380* \\
& (0.047) & (0.058) &  (0.026 ) & (0.077)\\
Chapters & -0.0075*& -0.0049* & -0.0002*& -0.0041* \\
& (0.000) & (0.000) & (8.01e-05) & ( 0.000) \\
Frequent relationship & 0.3765* & 0.4174* & 0.1881* & 0.6353*\\
& (0.006) & (0.007) & (0.013) & (0.010)\\
Author work count & -3.377e-05 * &-2.661e-05* & -5.21e-06* & -2.3e-05*\\
& (2.65e-06) & (3.28e-06) & (1.49e-06) & (4.33e-06)\\
Category (Female/Male) & -0.4615* & -0.2245* & -0.1142* & -0.3323*\\
& (0.006) & (0.007) & (0.009) & (0.009)\\
Category (Male/Male) & 0.1990* & 0.2637* & 0.1325* & 0.5224*\\
& (0.007) & (0.007) & (0.010) & (0.011)\\
Archive warnings (death) & -0.3719* & -0.2273* & -0.1432* & -0.4776*\\
& ( 0.010) & (0.012) & (0.010) & (0.017)\\
Rating (General audience) & -0.0339* & -0.1343* & -0.1163* & -0.6089*\\
& ( 0.008) & (0.010) & (0.007) & (0.015)\\
Rating (Mature) & -0.3559* & -0.4045* & -0.1171* & -0.4461*\\
& (0.008) & (0.009) & (0.009) & (0.012)\\
Fandom (Doctor Who) & -0.3892* & -0.3684* & -0.0969* & -0.5682*\\
& (0.006) & 0.008 & (0.007) & (0.010)\\
Fandom (Star Wars ) & 0.3575* & 0.2516* & 0.2331* & 0.2790*\\
& (0.016) & (0.020) & (0.016) & (0.026)\\
Age & 8.517e-05*& 0.0002* & -0.0001*& 1.296e-05*\\
& (4.53e-06) & (5.74e-06) & (9.09e-06 ) & (7.18e-06)\\
\hline 
Observations & 526992 & 526992 & 526992 & 526992 \\
Adjusted R-squared & 0.236 & 0.115 & 0.114  & 0.149\\
\bottomrule
\end{tabular}
 \begin{tablenotes}
      \small
      \item Standard errors are in parentheses.
      \item *p $<$ 0.001.
       \end{tablenotes}
\caption{OLS results}
\label{tab:regression}
\end{table*}%




\subsection*{Regression analysis}




Even when they seek out new fictions to read, they prefer to read things similar to what they have read before. This may be part of the nature of fan works --- because fans desire to see familiar characters and stories, it is reasonable for them to prefer fanfictions with familiar elements. The reasoning may extend to other parts of pop culture, 